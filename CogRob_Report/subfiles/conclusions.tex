%!TEX root = ../report.tex
\documentclass[../report.tex]{subfiles}

\begin{document}
\section{Conclusions}
\label{sec:conclusions}
LLM Cognitive System,
has memory,
has tools,
has mobile manipulator
simulation of household
task complition

\subsection{Summary}
\label{sec:conclusions:summary}
[DRAFT]
This project demonstrated how a large language model can serve as the cognitive core of an embodied agent in a simulated household environment. By embedding an LLM within a cognitive architecture that combines working memory, episodic memory, and a tool-calling interface, we enabled a mobile manipulator to perceive, plan, and act in a physically grounded setting. The agent successfully executed object manipulation and navigation tasks, such as placing items in a shelf and reordering them across locations. The results highlight that LLMs are capable of reasoning about spatial layouts, sequencing tool calls, and adapting strategies when faced with environmental constraints, though limitations such as task refusal and difficulty with multi-step planning remain.

\subsection{Contributions}
\label{sec:conclusions:contributions}
The main contributions of this work are:
\begin{itemize}
	\item Implementation of a PyBullet-based simulation of a mobile manipulator equipped with navigation and manipulation capabilities.
	\item Integration of an LLM into a simple cognitive architecture with memory and tool interfaces, allowing the agent to interact with its environment through perception, movement, and object manipulation.
	\item Empirical evaluation of two household tasks, which provided insights into the strengths and limitations of current LLMs when applied to embodied decision-making.
\end{itemize}

\subsection{Future Work}
\label{sec:conclusions:future_work}
[DRAFT]
While the current system demonstrates the feasibility of LLM-driven embodied agents, several challenges remain. Future work will focus on improving task generalization and robustness by refining memory integration and reducing hallucinations in task recognition. Automated evaluation pipelines could be implemented to generate quantitative performance benchmarks across tasks and models. Additionally, extending the simulation toward richer environments and incorporating more realistic grippers and perception pipelines would enable a closer approximation of real-world deployment. Finally, bridging the gap to physical robots remains a crucial next step for validating the scalability of this approach beyond simulation.
\end{document}
