%!TEX root = ../report.tex
\documentclass[../report.tex]{subfiles}

\begin{document}
    \section{Introduction}
    \label{sec:introduction}

    Large Language Models (LLMs) have been shown to exhibit impressive abilities in reasoning, language understanding, and task execution. This project explores how an LLM can serve as the core of a cognitive architecture embodied in a simulated 3D robot. By assigning the agent physical tasks in a realistic environment, we aim to investigate how such systems can perceive, plan, act, and adapt in grounded, interactive settings.

    \subsection{Motivation}
    \label{sec:introduction:motivation}

    To fully explore the cognitive capabilities of large language models, it is essential to situate them in environments that demand embodied, goal-directed interaction. This project investigates how LLMs function as the core of a cognitive architecture within a simulated 3D robotic setting, where the agent must physically manipulate objects and execute tasks in the real world.

    Unlike static text-based benchmarks, a robotic context enables examination of embodied reasoning, memory-guided action, and the capacity for grounded, anticipatory planning. By focusing on a single agent operating in a realistic environment -- such as a kitchen -- we can probe how LLMs reason about spatial layouts, interpret user instructions, and adapt plans based on feedback.

    Importantly, the robot's (and by extension, the LLM agent's) planning and decision-making capabilities take on a more grounded role, reasoning in natural language about the physical consequences of movements and manipulative actions. This opens new directions for exploring embodied cognition and could inform future developments in cognitive robotics, assistive AI, and multi-modal agent systems.

    \subsection{Problem Statement}
    \label{sec:introduction:problem_statement}

    Describe the problem you are addressing in the work.

    \subsection{Proposed Approach}
    \label{sec:introduction:proposed_approach}

    Write a short summary of your proposed approach.

\end{document}
